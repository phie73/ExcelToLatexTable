\documentclass[10pt]{article}
 
\title{Latex Tables}
\date{}
\author{sophia}

%packages
\usepackage{csvsimple}
\usepackage{hyperref}
\usepackage{booktabs}
\usepackage{tabularx}
 
\begin{document}
 
\maketitle
\tableofcontents
 
\section{csv simple}
\begin{itemize}
    \item requierd: package csvsimple
    \item \href{https://mirror.dogado.de/tex-archive/macros/latex/contrib/csvsimple/csvsimple.pdf}{doc csvsimple}
\end{itemize}
% \csvautotabular{csvfile.csv}
\csvautotabular{/home/sophia/ExcelToLatexTable/csvfile.csv}

\section{py pandas}
\begin{itemize}
    \item requierd: package booktabs
    \item \href{https://pandas.pydata.org/pandas-docs/stable/reference/api/pandas.DataFrame.to_latex.html}{pandas reference}
\end{itemize}
\begin{tabular}{llllll}
    \toprule
    {} &  a2 &  b2 &  c2 &  d2 &  e2 \\
    \midrule
    0 &  a3 &  b3 &  c3 &  d3 &  e3 \\
    1 &  a4 &  b4 &  c4 &  d4 &  e4 \\
    2 &  a5 &  b5 &  c5 &  d5 &  e5 \\
    3 &  a6 &  b6 &  c6 &  d6 &  e7 \\
    \bottomrule
    \end{tabular}

\section{default latex method}
\begin{center}
    \begin{tabular}{ c c c c c }
    a2 & b2 & c2 & d2 & e2 \\ 
    a3 & b3 & c3 & d3 & e3 \\ 
    a4 & b4 & c4 & d4 & e4 \\ 
    a5 & b5 & c5 & d5 & e5 \\ 
    a6 & b6 & c6 & d6 & e7 \\ 
    \end{tabular} 
\end{center}

\section{simple table}
\begin{center}
    \begin{tabular}{ | c| c| c| c| c | }
     \hline
    a2 & b2 & c2 & d2 & e2 \\ 
    a3 & b3 & c3 & d3 & e3 \\ 
    a4 & b4 & c4 & d4 & e4 \\ 
    a5 & b5 & c5 & d5 & e5 \\ 
    a6 & b6 & c6 & d6 & e7 \\ 
    \hline
    \end{tabular}
\end{center}

\section{simple table 2.0}
\begin{center}
    \begin{tabular}{ | c| c| c| c| c| }
     \hline
    a2 & b2 & c2 & d2 & e2 \\ 
     \hline 
    a3 & b3 & c3 & d3 & e3 \\ 
     \hline 
    a4 & b4 & c4 & d4 & e4 \\ 
     \hline 
    a5 & b5 & c5 & d5 & e5 \\ 
     \hline 
    a6 & b6 & c6 & d6 & e7 \\ 
     \hline 
    \end{tabular}
\end{center}

\section{swe table}

package "tabularx" requierd \\
originally created for our software engineering project
\begin{table}[h]
    \centering
    \tiny
    \begin{tabularx}{\textwidth}{|l|l|X|l|l|} 
    \hline 
    a2 & b2 & c2 & d2 & e2 \\ 
    \hline
    a3 & b3 & c3 & d3 & e3 \\ 
    \hline
    a4 & b4 & c4 & d4 & e4 \\ 
    \hline
    a5 & b5 & c5 & d5 & e5 \\ 
    \hline
    a6 & b6 & c6 & d6 & e7 \\ 
    \hline
    \end{tabularx}
\end{table}

\end{document}